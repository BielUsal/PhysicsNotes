%Every piece of package I've acumulated over the last years
%%%%%%%%%%%%%%%%%%%%%%%%%%%%%%%%%%%%%%%%%%%%%%%%%%%%%%%%%%%%%%%%%%%%%%%%%%%%%%%%%%%%%%%%%%%%%%
\documentclass[a4paper,12pt]{article}
\usepackage[utf8]{inputenc}
\usepackage{imakeidx}
\usepackage{graphicx}
\usepackage{float}
\usepackage{amsmath}
\usepackage[backend=bibtex,style=verbose]{biblatex}
\bibliography{bibliography}
\usepackage{csquotes}
\usepackage{tcolorbox}
\usepackage{multirow}
\usepackage{caption}
\usepackage{afterpage}
\usepackage[margin=1in]{geometry}
\usepackage[english,spanish]{babel}
\usepackage{tikz}
\usepackage{mwe}
\usepackage{circuitikz}
\usepackage{subcaption}
%%%%%%%%%%%%%%%%%%%%%%%%%%%%%%%%%%%%%%%%%%%%%%%%%%%%%%%%%%%%%%%%%%%%%%%%%%%%%%%%%%%%%%%%%%%%%%
\begin{document}
\title{Evaluación Continua de Mecánica II\\ Temas 2-3}
\author{Gabriel D'Andrade Furlanetto}
\maketitle 

\section{Péndulo de Foucault}

\subsection*{a) Escribe la segunda ley de Newton para el grave y demuestra que la tensión $\boldsymbol{T}$ se puede aproximar como $T \approx m\boldsymbol{g}_{ef}$, donde $\boldsymbol{g}_{ef}$ es la gravedad efectiva.}

Como estamos en el sistema de la Tierra, sabemos que podemos escribir la segunda Ley de Newton para este sistema como:

\begin{equation}
  m\ddot{\boldsymbol{r}} = \boldsymbol{F}_{ap} + \boldsymbol{F}_{cf} + \boldsymbol{F}_{cor}
\end{equation}

Para nuestro problema en particular, la fuerza aplicada será, por un lado, la tensión y, por otro, la gravedad. De esta manera, tendremos que:

\begin{equation}
  m \ddot{\boldsymbol{r}} = \boldsymbol{T} + m\boldsymbol{g} + m(\boldsymbol{\omega} \times \boldsymbol{r}) + 2m (\dot{\boldsymbol{r}} \times \boldsymbol{\omega})
\end{equation}

Esta ecuación se puede simplificar bastante si consideramos que el término centrífugo se puede combinar al gravitatorio y, al final, tendremos un término de la gravedad efectiva\footnote{Tomaremos esto simplemente como un hecho conocido, pero cualquier libro texto estándar de mecánica lo tiene tratado explícitamente.}:
\begin{equation}
  m \ddot{\boldsymbol{r}} = \boldsymbol{T} + m\boldsymbol{g}_ef + 2m (\dot{\boldsymbol{r}} \times \boldsymbol{\omega})
\end{equation}


\section{Cono con Canica}


\end{document}

